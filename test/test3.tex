1. Informal and Colloquial Expressions
\section{Introduction}
The study focused on America as the primary region of interest. The results were bad and showed significant variation. A big, humongous, and huge amount of evidence supports this theory. The senator said that the government has got to act. The writer got a prize for her work, and Table 5.2 gives evidence to support this conclusion. The findings were good. The scholar kind of, sort of agreed with the hypothesis. The study will run from March till May. The interviews showed a shared concern over safety.
2. Vague Writing
\section{Methods}
The experiment required a bit of catalyst. A couple of students were surveyed, and some provided additional feedback. A lot of attention was given to certain results, while lots of minor details were overlooked. The government spent a ton of money on the project, and civil engineers must consider load, terrain, weather and so on when designing a bridge. The writer could be referencing anything or something else. It was a nice painting. Most scientists support the measure. The report included stuff and thing that were not clearly defined.
3. Exaggerations
\section{Results}
The results were always the same. The findings definitely, absolutely provide significant support for the proposed legislation. Every recorded species of this genus is found in the Brazilian Amazon. Surveyed respondents never chose the fourth option. None of the respondents answered positively. This finding proves the hypothesis, and the proof is clear.
4. Subjectivity
\section{Discussion}
The building was beautiful, and the composition was wonderful, awful, ugly, and hideous. The candidate had a better plan to address climate change than his opponent. Clearly, naturally, of course, obviously, and undoubtedly, the results indicate a trend. The applicant was the perfect, ideal, best candidate. Activists should study the protest.
5. Clichés and Colloquialisms
\section{Conclusion}
The two sides reached a happy medium. The proposal encountered a stumbling block when the opposing party mounted a negative press campaign. Unlike previous proposals, the resolution was above board. At the end of the day, when all is said and done, the new law did not impact the rural population. The researcher had to get through multiple texts. In this day and age, in recent years, social media use has become prevalent amongst millennials. The economist was known to think outside the box.
6. Fillers
\section{Supplementary Notes}
The scientist had literally explored every option. The results were really too revealing. This quote serves to helps to illustrate the author’s primary argument. The poem is so interesting because it uses an unusual rhyme pattern. Her testimony was very, extremely crucial.
7. First or Second Person
\section{Author's Perspective}
We performed a regression analysis. My research indicates a significant trend. Our data confirms the hypothesis. You might think your assessment is correct, but your conclusion should be based on evidence.
8. Gendered Language
\section{Language Awareness}
The wheel is one of mankind’s greatest inventions. The policeman and congresswoman attended the meeting, along with the usherette and actress. The weather girl and lady reporter warned viewers of the approaching storm.
9. Miscellaneous
\section{Miscellaneous Academic Advice}
Me and myself are rarely used in formal writing. The researcher will also discuss the results. Everyone should be aware that must is a demanding word. Currently, today, many believe that he, she, we, they, their, them, or it can be omitted for clarity. Thus, therefore, however, otherwise, and this are often filler words. Think and feel should be supported by evidence.
10. Miscellaneous-II
\section{Common Pitfalls}
The report included stuff and things. This is sort of, kind of problematic. A bunch of samples and tons of data were collected. Lots of problems etc. were identified. Participants may submit essays and/or reports. Basically, quite, just a little, a few, maybe, perhaps, a number of, huge, massive, amazing, incredible, cool, neat, awesome, bad, nice, super, mega, a bit, a lot, a little bit, somehow, totally, always, never, got, gonna, wanna, okay, ok, anyways, plus, big, small, nowadays, kids, guy, guys, folks, clearly, obviously, definitely, everyone knows, it is obvious that, there is no doubt, without a doubt, it goes without saying, can't, don't, won't, isn't, doesn't, ain't, gotta, kinda, sorta, TV, info, pics, LOL, ASAP, FYI, In conclusion, To sum up, As we all know — all these words and phrases should be avoided in academic writing.

small
aaa \small bbda

\big